% Options for packages loaded elsewhere
\PassOptionsToPackage{unicode}{hyperref}
\PassOptionsToPackage{hyphens}{url}
%
\documentclass[
  portrait]{article}
\usepackage{amsmath,amssymb}
\usepackage{iftex}
\ifPDFTeX
  \usepackage[T1]{fontenc}
  \usepackage[utf8]{inputenc}
  \usepackage{textcomp} % provide euro and other symbols
\else % if luatex or xetex
  \usepackage{unicode-math} % this also loads fontspec
  \defaultfontfeatures{Scale=MatchLowercase}
  \defaultfontfeatures[\rmfamily]{Ligatures=TeX,Scale=1}
\fi
\usepackage{lmodern}
\ifPDFTeX\else
  % xetex/luatex font selection
\fi
% Use upquote if available, for straight quotes in verbatim environments
\IfFileExists{upquote.sty}{\usepackage{upquote}}{}
\IfFileExists{microtype.sty}{% use microtype if available
  \usepackage[]{microtype}
  \UseMicrotypeSet[protrusion]{basicmath} % disable protrusion for tt fonts
}{}
\makeatletter
\@ifundefined{KOMAClassName}{% if non-KOMA class
  \IfFileExists{parskip.sty}{%
    \usepackage{parskip}
  }{% else
    \setlength{\parindent}{0pt}
    \setlength{\parskip}{6pt plus 2pt minus 1pt}}
}{% if KOMA class
  \KOMAoptions{parskip=half}}
\makeatother
\usepackage{xcolor}
\usepackage[margin=1in]{geometry}
\usepackage{longtable,booktabs,array}
\usepackage{calc} % for calculating minipage widths
% Correct order of tables after \paragraph or \subparagraph
\usepackage{etoolbox}
\makeatletter
\patchcmd\longtable{\par}{\if@noskipsec\mbox{}\fi\par}{}{}
\makeatother
% Allow footnotes in longtable head/foot
\IfFileExists{footnotehyper.sty}{\usepackage{footnotehyper}}{\usepackage{footnote}}
\makesavenoteenv{longtable}
\usepackage{graphicx}
\makeatletter
\def\maxwidth{\ifdim\Gin@nat@width>\linewidth\linewidth\else\Gin@nat@width\fi}
\def\maxheight{\ifdim\Gin@nat@height>\textheight\textheight\else\Gin@nat@height\fi}
\makeatother
% Scale images if necessary, so that they will not overflow the page
% margins by default, and it is still possible to overwrite the defaults
% using explicit options in \includegraphics[width, height, ...]{}
\setkeys{Gin}{width=\maxwidth,height=\maxheight,keepaspectratio}
% Set default figure placement to htbp
\makeatletter
\def\fps@figure{htbp}
\makeatother
\setlength{\emergencystretch}{3em} % prevent overfull lines
\providecommand{\tightlist}{%
  \setlength{\itemsep}{0pt}\setlength{\parskip}{0pt}}
\setcounter{secnumdepth}{5}
\renewcommand{\figurename}{Supplementary Figure }
\renewcommand{\tablename}{Supplementary Table}
\makeatletter
\def\fnum@figure{\figurename\thefigure}
\makeatother
\usepackage{booktabs}
\usepackage{longtable}
\usepackage{array}
\usepackage{multirow}
\usepackage{wrapfig}
\usepackage{float}
\usepackage{colortbl}
\usepackage{pdflscape}
\usepackage{tabu}
\usepackage{threeparttable}
\usepackage{threeparttablex}
\usepackage[normalem]{ulem}
\usepackage{makecell}
\usepackage{xcolor}
\ifLuaTeX
  \usepackage{selnolig}  % disable illegal ligatures
\fi
\usepackage{bookmark}
\IfFileExists{xurl.sty}{\usepackage{xurl}}{} % add URL line breaks if available
\urlstyle{same}
\hypersetup{
  hidelinks,
  pdfcreator={LaTeX via pandoc}}

\author{}
\date{\vspace{-2.5em}}

\begin{document}

\begin{table}
\centering
\caption{\label{tab:tbl-pars-stan}Correspondence between parameters named in the manuscript and variables in the associated code (publicly available on GitHub).}
\centering
\fontsize{8}{10}\selectfont
\begin{tabu} to \linewidth {>{\raggedright}X>{\centering}X>{\centering}X}
\toprule
Description & Paper & Code\\
\midrule
\cellcolor{gray!10}{Recorded density} & \cellcolor{gray!10}{$D_{p, t}$} & \cellcolor{gray!10}{dens}\\
SD of density at the log-scale & $\sigma_{\mathrm{obs}}$ & sigma_obs\\
\cellcolor{gray!10}{Natural mortality rate} & \cellcolor{gray!10}{$m$} & \cellcolor{gray!10}{m}\\
Fishing mortality rate & $f_{a, t}$ & f\\
\cellcolor{gray!10}{Intercept for detection probability} & \cellcolor{gray!10}{$\beta_0$} & \cellcolor{gray!10}{beta_obs_int}\\
\addlinespace
Coefficient linking detection probability to density & $\beta_1$ & beta_obs\\
\cellcolor{gray!10}{Overall average recruitment} & \cellcolor{gray!10}{$\mu$} & \cellcolor{gray!10}{mean_recruits}\\
Conditional variance in the AR(1) process & $\sigma_{\mathrm{proc}}$ & sigma_r\\
\cellcolor{gray!10}{White noise used in the AR(1) process} & \cellcolor{gray!10}{$z_t$} & \cellcolor{gray!10}{raw}\\
Autcorrelation from AR(1) term & $\alpha$ & alpha\\
\addlinespace
\cellcolor{gray!10}{Excess in natural mortality due to temperature} & \cellcolor{gray!10}{$\gamma$} & \cellcolor{gray!10}{m_e}\\
Isotropic dispersal rate & $\delta$ & d\\
\cellcolor{gray!10}{Sea bottom temperature} & \cellcolor{gray!10}{$T_{p, t}$} & \cellcolor{gray!10}{sbt}\\
How much tax per unit of temperature & $\beta_{tax}$ & beta_t\\
\cellcolor{gray!10}{Suitability index} & \cellcolor{gray!10}{$I_{p, t}$} & \cellcolor{gray!10}{T_adjust}\\
\addlinespace
Temperature optimizing $I_{p, t}$ & $\tau$ & Topt\\
\cellcolor{gray!10}{Width parameter from $I_{p, t}$} & \cellcolor{gray!10}{$\omega$} & \cellcolor{gray!10}{width}\\
\bottomrule
\end{tabu}
\end{table}

\begin{table}
\centering
\caption{\label{tab:priors}Hyperparameters used in the DRM.}
\centering
\fontsize{8}{10}\selectfont
\begin{tabu} to \linewidth {>{\raggedright}X>{\raggedright}X}
\toprule
Parameter & Prior.Distribution\\
\midrule
\cellcolor{gray!10}{$\beta_{obs,1}$} & \cellcolor{gray!10}{Normal $\sim (0.001, 0.1)$}\\
$\beta_{obs,0}$ & Normal $\sim (-100, 4)$\\
\cellcolor{gray!10}{$z$} & \cellcolor{gray!10}{Normal $\sim (0, 1)$}\\
$\sigma_{proc}$ & Normal $\sim (0.2, 0.1)$\\
\cellcolor{gray!10}{$\sigma_{obs}$} & \cellcolor{gray!10}{Normal $\sim (0.21, 0.2)$}\\
\addlinespace
$width$ & Normal $\sim (4, 2)$\\
\cellcolor{gray!10}{$T_{opt}$} & \cellcolor{gray!10}{Normal $\sim (18, 2)$}\\
$d$ & Normal $\sim (0.01, 0.1)$\\
\cellcolor{gray!10}{$\beta_{tax}$} & \cellcolor{gray!10}{Normal $\sim (0, 2)$}\\
$\alpha$ & Beta $\sim (12, 20)$\\
\addlinespace
\cellcolor{gray!10}{$\mu_r$} & \cellcolor{gray!10}{Lognormal $\sim (7, 5)$}\\
$m_e$ & Exponential $\sim 2.3$\\
\bottomrule
\end{tabu}
\end{table}

\begin{table}
\centering
\caption{\label{tab:btemp-train}Linear regressions of sea bottom temperature (measured in the trawl surveys) on year within each patch during the model training interval (1972-2006). Values are rounded to three digits.}
\centering
\fontsize{8}{10}\selectfont
\begin{tabu} to \linewidth {>{\raggedleft}X>{\raggedleft}X>{\raggedleft}X>{\raggedleft}X}
\toprule
Patch & Estimate & Standard Error & P-value\\
\midrule
\cellcolor{gray!10}{35} & \cellcolor{gray!10}{0.002} & \cellcolor{gray!10}{0.040} & \cellcolor{gray!10}{0.956}\\
36 & 0.045 & 0.023 & 0.049\\
\cellcolor{gray!10}{37} & \cellcolor{gray!10}{0.056} & \cellcolor{gray!10}{0.022} & \cellcolor{gray!10}{0.012}\\
38 & 0.030 & 0.017 & 0.089\\
\cellcolor{gray!10}{39} & \cellcolor{gray!10}{0.047} & \cellcolor{gray!10}{0.013} & \cellcolor{gray!10}{0.000}\\
\addlinespace
40 & 0.019 & 0.006 & 0.001\\
\cellcolor{gray!10}{41} & \cellcolor{gray!10}{0.027} & \cellcolor{gray!10}{0.008} & \cellcolor{gray!10}{0.001}\\
42 & -0.003 & 0.004 & 0.493\\
\cellcolor{gray!10}{43} & \cellcolor{gray!10}{0.003} & \cellcolor{gray!10}{0.005} & \cellcolor{gray!10}{0.479}\\
44 & 0.006 & 0.010 & 0.546\\
\bottomrule
\end{tabu}
\end{table}

\begin{table}
\centering
\caption{\label{tab:btemp-test}Linear regressions of sea bottom temperature (measured in the trawl surveys) on year within each patch during the model testing interval (2007-2016). Values are rounded to three digits.}
\centering
\fontsize{8}{10}\selectfont
\begin{tabu} to \linewidth {>{\raggedleft}X>{\raggedleft}X>{\raggedleft}X>{\raggedleft}X}
\toprule
Patch & Estimate & Standard Error & P-value\\
\midrule
\cellcolor{gray!10}{35} & \cellcolor{gray!10}{0.159} & \cellcolor{gray!10}{0.235} & \cellcolor{gray!10}{0.500}\\
36 & 0.226 & 0.176 & 0.202\\
\cellcolor{gray!10}{37} & \cellcolor{gray!10}{0.161} & \cellcolor{gray!10}{0.153} & \cellcolor{gray!10}{0.295}\\
38 & -0.042 & 0.111 & 0.708\\
\cellcolor{gray!10}{39} & \cellcolor{gray!10}{0.089} & \cellcolor{gray!10}{0.090} & \cellcolor{gray!10}{0.321}\\
\addlinespace
40 & 0.360 & 0.044 & 0.000\\
\cellcolor{gray!10}{41} & \cellcolor{gray!10}{0.158} & \cellcolor{gray!10}{0.059} & \cellcolor{gray!10}{0.007}\\
42 & 0.168 & 0.029 & 0.000\\
\cellcolor{gray!10}{43} & \cellcolor{gray!10}{0.174} & \cellcolor{gray!10}{0.035} & \cellcolor{gray!10}{0.000}\\
44 & 0.253 & 0.045 & 0.000\\
\bottomrule
\end{tabu}
\end{table}

\begin{figure}
\centering
\includegraphics{09_supplement_files/figure-latex/frequency-distribution-1.pdf}
\caption{\label{fig:frequency-distribution}Frequency distribution of summer flounder abundance in all hauls used in the analysis (testing and training data combined; \emph{n} = 12,203).}
\end{figure}

\begin{figure}
\centering
\includegraphics{09_supplement_files/figure-latex/haul-distribution-1.pdf}
\caption{\label{fig:haul-distribution}Number of hauls per year; \emph{n} = 12,203.}
\end{figure}

\begin{figure}
\centering
\includegraphics{09_supplement_files/figure-latex/null-DRM-by-patch-1.pdf}
\caption{\label{fig:null-DRM-by-patch}Median density estimated by the null DRM in the training dataset by patch and year (black line). Blue shading represents the 50\%, 80\%, and 95\% credible intervals. Red points are the observed data. Note that y-axes vary by patch.}
\end{figure}

\begin{figure}
\centering
\includegraphics{09_supplement_files/figure-latex/T-recruit-DRM-by-patch-1.pdf}
\caption{\label{fig:T-recruit-DRM-by-patch}Median density estimated by the temperature-dependent recruitment DRM in the training dataset by patch and year (black line). Blue shading represents the 50\%, 80\%, and 95\% credible intervals. Red points are the observed data. Note that y-axes vary by patch.}
\end{figure}

\begin{figure}
\centering
\includegraphics{09_supplement_files/figure-latex/T-mortality-DRM-by-patch-1.pdf}
\caption{\label{fig:T-mortality-DRM-by-patch}Median density estimated by the temperature-dependent mortality DRM in the training dataset by patch and year (black line). Blue shading represents the 50\%, 80\%, and 95\% credible intervals. Red points are the observed data. Note that y-axes vary by patch.}
\end{figure}

\begin{figure}
\centering
\includegraphics{09_supplement_files/figure-latex/T-movement-DRM-by-patch-1.pdf}
\caption{\label{fig:T-movement-DRM-by-patch}Median density estimated by the temperature-dependent movement DRM in the training dataset by patch and year (black line). Blue shading represents the 50\%, 80\%, and 95\% credible intervals. Red points are the observed data. Note that y-axes vary by patch.}
\end{figure}

\begin{figure}
\centering
\includegraphics{09_supplement_files/figure-latex/posterior-plot-1.pdf}
\caption{\label{fig:posterior-plot}Posterior distributions of parameters from the four fitted DRMs. These parameters are d, the annual dispersal fraction between adjacent patches; mean\_recruits, the average density of recruits per patch; sigma\_obs, the standard deviation of density at the log scale; Topt, the optimal temperature estimated for whichever process (movement or recruitment or mortality) was estimated as temperature-dependent, in °C; and width, a parameter controlling how sensitive the modeled process was to temperature. See Supp. Tab. 1 for correspondence of code variables to model parameters in the manuscript.}
\end{figure}

\begin{figure}
\centering
\includegraphics{09_supplement_files/figure-latex/resid-by-yr-1.pdf}
\caption{\label{fig:resid-by-yr}Residuals (estimated minus actual values) of range metrics each year over the ten-year testing period for the focal models in the main text: DRM with no temperature effect (DRM null) or a temperature effect on recruitment, mortality, or movement; a GAM SDM; and a persistence forecast. The horizontal dashed line indicates zero residuals, i.e., the estimate perfectly matches the testing data.}
\end{figure}

\end{document}
